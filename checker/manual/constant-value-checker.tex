\htmlhr
\chapter{Constant Value Checker\label{constant-value-checker}}

The Constant Value Checker is a constant propagation analysis: for
each variable, it determines whether that variable's value can be
known at compile time.

There are two ways to run the Constant Value Checker.
\begin{itemize}
\item
Typically, it is automatically run by another type checker.
When using the Constant Value Checker as part of another checker, the
\code{statically-executable.astub} file in the Constant Value Checker directory must
be passed as a stub file for the checker.
\item
Alternately, you can run just the Constant Value Checker, by
supplying the following command-line option to javac:
\code{-processor org.checkerframework.common.value.ValueChecker -Astubs=statically-executable.astub}
\end{itemize}

\section{Annotations\label{constant-value-checker-annotations}}

The Constant Value Checker uses type annotations to indicate the value of
an expression (Section~\ref{constant-value-checker-type-annotations}), and
it uses method annotations to indicate methods that the Constant Value
Checker can execute at compile time
(Section~\ref{staticallyexecutable-annotation}).


\subsection{Type Annotations\label{constant-value-checker-type-annotations}}

Typically, the programmer does not write any type annotations.  Rather, the
type annotations are inferred by the Constant Value Checker.
The programmer is also permitted to write type annotations.  This is only necessary in
locations where the Constant Value Checker does not infer annotations:  on fields
and method signatures.

The type annotations are
\refqualclass{common/value/qual}{BoolVal},
\refqualclass{common/value/qual}{IntVal},
\refqualclass{common/value/qual}{DoubleVal}, and
\refqualclass{common/value/qual}{StringVal}.

Each type annotation takes as an argument a set of values, and its meaning
is that at run time, the expression evaluates to one of the values.  For
example, an expression of type
\<\refqualclass{common/value/qual}{StringVal}("a", "b")> evaluates to
one of the values \<"a">, \<"b">, or \<null>.
The set is limited to 10 entries; if a variable
could be more than 10 different values, the Constant Value
Checker gives up and its type becomes
\refqualclass{common/value/qual}{UnknownVal} instead.

% \refqualclass{checker/value/qual}{BottomVal}, meaning that the expression
% is dead or always has the value \<null>.

Figure~\ref{fig-value-hierarchy} shows the
subtyping relationship among the type annotations.
For two annotations of the same type, subtypes have a smaller set of
possible values, as also shown in the figure.
Because \<int> can be casted to \<double>, an \<@IntVal> annotation is a
subtype of a \<@DoubleVal> annotation with the same values.

\begin{figure}
\includeimage{value-subtyping}{2.75cm}
\caption{The type qualifier hierarchy of the Constant Value Checker
annotations. Qualifiers in gray are used
internally by the type system but should never be written by a
programmer.  On the right are examples of additional subtyping
relationships that depend on the annotations' arguments.}
\label{fig-value-hierarchy}
\end{figure}

Figure~\ref{fig-value-multivalue} shows how the Constant Value Checker
infers type annotations (using flow-sensitive type qualifier refinement, Section~\ref{type-refinement}).

\begin{figure}
\begin{Verbatim}
public void foo(boolean b) {
    int i = 1;     // i has type:  @IntVal({1}) int
    if (b) {
        i = 2;     // i now has type:  @IntVal({2}) int
    }
                   // i now has type:  @IntVal({1,2}) int
    i = i + 1;     // i now has type:  @IntVal({2,3}) int
}
\end{Verbatim}
\caption{The Constant Value Checker infers different types
  for a variable on different lines of the program.}
\label{fig-value-multivalue}
\end{figure}


\subsection{Compile-time execution of expressions\label{staticallyexecutable-annotation}}

Whenever all the operands of an expression are compile-time constants (that
is, their types have constant-value type annotations), the Constant Value
Checker attempts to execute the expression.  This is independent of any
optimizations performed by the compiler and does not affect the code that
is generated.

The Constant Value Checker statically executes operators that do
not throw exceptions (e.g., \<+>, \<->, \code{<\relax<}, \<!=>), and also
calls to methods annotated with
\refqualclass{common/value/qual}{StaticallyExecutable}.

\begin{figure}
\begin{Verbatim}
@StaticallyExecutable @Pure
public int foo(int a, int b) {
    return a + b;
}

public void bar() {
    int a = 5;          // a has type:  @IntVal({5}) int
    int b = 4;          // b has type:  @IntVal({4}) int
    int c = foo(a, b);  // c has type:  @IntVal({9}) int
}
\end{Verbatim}
\caption{The
  \refqualclass{common/value/qual}{StaticallyExecutable} annotation enables
  constant propagation through method calls.}
\label{fig-staticallyexecutable}
\end{figure}

A \<@StaticallyExecutable> method must
be \refqualclass{dataflow/qual}{Pure} (side-effect-free and
deterministic).
Additionally, a \<@StaticallyExecutable> method and any method it calls must be on
the classpath for the compiler, because they are reflectively called at
compile-time to perform the constant value analysis. Any standard
library methods (such as those annotated as \<@StaticallyExecutable> in file
\<statically-executable.astub>) will already be on the
classpath.

To use \<@StaticallyExecutable> on methods in your own code, you should
first compile the code without the Constant Value Checker and then add
the location of the resulting \code{.class} files to the
classpath. This can be done by either adding the destination path to
your environment variable \code{CLASSPATH} or by passing the
argument \code{-classpath path/to/class/files} to the call. The latter
would look similar to:
\code{-processor org.checkerframework.common.value.ValueChecker
-Astubs=statically-executable.astub -classpath \$CLASSPATH:\$MY\_PROJECT/build/}



\section{Warnings\label{value-checker-warnings}}

The Constant Value Checker issues a warning if it cannot load and run, at
compile time, a method marked as \<@StaticallyExecutable>.  If it issues
such a warning, then the return value of the method will be \<@UnknownVal>
instead of being able to be resolved to a specific value annotation.
Some examples of these:

\begin{itemize}
\item \code{[class.find.failed] Failed to find class named Test.}

  The checker could not find the class
  specified for resolving a \<@StaticallyExecutable> method. Typically
  this is caused by not providing the path of a class-file needed to
  the classpath.

\item \code{[method.find.failed] Failed to find a method named foo with argument types [@IntVal(3) int]. Treating result as @UnknownVal}

  The checker could not find the method \code{foo(int)} specified for
  resolving a \<@StaticallyExecutable> method, but could find the
  class. This is usually due to providing an outdated version of the
  class-file that does not contain the \<@StaticallyExecutable>
  method.

\item \code{[method.evaluation.exception] Failed to evaluate method public static int Test.foo(int) because it threw an exception: java.lang.ArithmeticException: / by zero. Treating result as @UnknownVal}

  An exception was thrown when trying to statically execute the
  method. In this case it was a divide-by-zero exception. If the
  arguments to the method each only had one value in their annotations
  then this exception will always occur when the program is actually
  run as well. If there are multiple possible values then the exception
  might not be thrown on every execution, depending on the run-time values.

\end{itemize}

There is one other situation in which the Constant Value Checker produces a
warning message:

\begin{itemize}
\item \code{[too.many.values.given] Annotation ignored because the maximum
  number of values tracked is 10.}

  The Constant Value Checker only tracks up to 10 possible values for an
  expression.  If you write an annotation with more values than will be
  tracked, the annotation is ignored.

\end{itemize}


%%  LocalWords:  9cm UnknownVal 5cm StringValue BottomVal 9cm 5cm astub
%  LocalWords:  StaticallyExecutable BoolVal IntVal DoubleVal StringVal
%%  LocalWords:  3cm 75cm classpath
