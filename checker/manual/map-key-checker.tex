\htmlhr
\chapter{Map Key Checker\label{map-key-checker}}

The Map Key Checker tracks which values are keys for which maps.  If variable
\code{v} has type \code{@KeyFor("m")...}, then the value of \code{v} is a key
in Map \code{m}.  That is, the expression \code{m.containsKey(v)} evaluates to
\code{true}.

Section~\ref{map-key-qualifiers} describes how \code{@KeyFor} annotations
enable the
Nullness Checker (\chapterpageref{nullness-checker}) to treat calls to
\sunjavadoc{java/util/Map.html\#get-java.lang.Object-}{\code{Map.get}}
more precisely by refining its result to \<@NonNull> in some cases.

You will not typically run the Map Key Checker.  It is automatically run by
other checkers, in particular the Nullness Checker.

You can suppress warnings related to map keys with
\<@SuppressWarnings("keyfor")>; see \chapterpageref{suppressing-warnings}.

\section{Map key annotations\label{map-key-annotations}}

These qualifiers are part of the Map Key type system:

\begin{description}

\item[\refqualclasswithparams{checker/nullness/qual}{KeyFor}{String[] maps}]
  indicates that the value assigned to the annotated variable is a key for at
  least the given maps.

\item[\refqualclass{checker/nullness/qual}{UnknownKeyFor}]
  is used internally by the type system but should never be written by a
  programmer.  It indicates that the value assigned to the annotated
  variable is not known to be a key for any map.  It is the default type
  qualifier.
  
\item[\refqualclass{checker/nullness/qual}{KeyForBottom}]
  is used internally by the type system but should never be written by a
  programmer.  

\end{description}  

\begin{figure}
\includeimage{map-key-keyfor}{5cm}
\caption{The subtyping relationship of the Map Key Checker's qualifiers.
\<@KeyFor(A)> is a supertype of \<@KeyFor(B)> if and only if \<A> is a subset of
\<B>.  Qualifiers in gray are used internally by the type system but should
never be written by a programmer.}
\label{fig-map-key-keyfor-hierarchy}
\end{figure}

\section{Examples\label{map-key-examples}}

The Map Key Checker keeps track of which variables reference keys to
which maps.  A variable annotated with \<@KeyFor(\emph{mapSet})> can only
contain a value that is a key for all the maps in \emph{mapSet}.  For example:

\begin{verbatim}
Map<String,Date> m, n;
@KeyFor("m") String km;
@KeyFor("n") String kn;
@KeyFor({"m", "n"}) String kmn;
km = kmn;   // OK - a key for maps m and n is also a key for map m
km = kn;    // error: a key for map n is not necessarily a key for map m
\end{verbatim}


As with any annotation, use of the \<@KeyFor> annotation may force you to
slightly refactor your code.  For example, this would be illegal:

\begin{verbatim}
Map<String,Object> m;
Collection<@KeyFor("m") String> coll;
coll.add(x);   // error:  coll's element type is @KeyFor("m") String, but x does not have that type
m.put(x, ...);
\end{verbatim}

\noindent
The example type-checks if you reorder the two calls:

\begin{verbatim}
Map<String,Object> m;
Collection<@KeyFor("m") String> coll;
m.put(x, ...);    // after this statement, x has type @KeyFor("m") String
coll.add(x);      // OK
\end{verbatim}



\section{Inference of @KeyFor annotations\label{map-key-annotations-inference}}

Within a method body, you usually do not have to write \<@KeyFor> explicitly,
because the checker infers it based on usage patterns.  When the Map Key
Checker encounters a run-time check for map keys, such as
``\<if (m.containsKey(k)) ...>'', then the Map Key Checker refines the type of
\<k> to \<@KeyFor("m")> within the scope of the test (or until \<k> is
side-effected within that scope).  The Map Key Checker also infers \<@KeyFor>
annotations based on iteration over a map's
\sunjavadoc{java/util/Map.html\#keySet--}{\textrm{key set}} or calls to
\sunjavadoc{java/util/Map.html\#put-K-V-}{put}
or
\sunjavadoc{java/util/Map.html\#containsKey-java.lang.Object-}{containsKey}.
For more details about type refinement, see Section~\ref{type-refinement}.

Suppose we have these declarations:

\begin{verbatim}
Map<String,Date> m = new Map<String,Date>();
String k = "key";
@KeyFor("m") String km;
\end{verbatim}

Ordinarily, the following assignment does not type-check:

\begin{verbatim}
km = k;   // Error since k is not known to be a key for map m.
\end{verbatim}

The following examples show cases where the Map Key Checker
infers a \<@KeyFor> annotation for variable \<k> based on usage patterns,
enabling the \<km = k> assignment to type-check.


\begin{verbatim}
m.put(k, ...);
// At this point, the type of k is refined to @KeyFor("m") String.
km = k;   // OK


if (m.containsKey(k)) {
    // At this point, the type of k is refined to @KeyFor("m") String.
    km = k;   // OK
    ...
}
else {
    km = k;   // Error since k is not known to be a key for map m.
    ...
}
\end{verbatim}


The following example shows a case where the Map Key Checker resets its
assumption about the type of a field used as a key because that field may have
been side-effected.

\begin{verbatim}
class MyClass {
    private Map<String,Object> m;
    private String k;   // The type of k defaults to @UnknownKeyFor String
    private @KeyFor("m") String km;

    public void myMethod() {
        if (m.containsKey(k)){
            km = k;   // OK: the type of k is refined to @KeyFor("m") String

            sideEffectFreeMethod();
            km = k;   // OK: the type of k is not affected by the method call
                      // and remains @KeyFor("m") String

            otherMethod();
            km = k;   // error: At this point, the type of k is once again
                      // @UnknownKeyFor String, because otherMethod might have
                      // side-effected k such that it is no longer a key for map m.
        }
    }

    @SideEffectFree
    private void sideEffectFreeMethod() { ... }

    private void otherMethod() { ... }
}
\end{verbatim}


\subsection{When local variable inference is not possible\label{map-key-annotations-inference-exceptions}}

One usage pattern where you \emph{do} have to write \<@KeyFor> is for a
collection that you know to be a subset of a key set.  In the example
below, the Map Key Checker cannot verify that the set returned by the call to \<nounSubset> in
\<outputAllNounDefinitions> contains only strings that are keys for \<dict>.

\begin{verbatim}
Map<String, String> dict;   // English dictionary. Maps words to their definitions.

// Method nounSubset's declaration uses no @KeyFor annotations
// because in addition to being used by the dictionary feature,
// it is also used by a spell checker that only stores sets of words
// and does not use the notions of dictionaries, maps, or keys.
public Set<String> nounSubset(Set<String> words) { ... }

void outputAllNounDefinitions() {
    @SuppressWarnings("keyfor") // argument elements are @KeyFor("dict"), so result elements are too
    Set<@KeyFor("dict") String> nounSet = nounSubset(dict.keySet());

    for (@KeyFor("dict") String noun : nounSet) {
      String definition = dict.get(noun);
      System.out.println(noun + ": " + definition);
    }
}
\end{verbatim}

For this pattern, it would be preferable to annotate the parameter and return
values of \<nounSubset> with \<@PolyAll> (Section~\ref{polyall}) instead.
The Checker Framework does not currently support \<@PolyAll> for the
Map Key Checker and the Checker Framework developers are working to
correct this --- it is tracked by
\href{http://code.google.com/p/checker-framework/issues/detail?id=429}{Issue 429}.

%%  LocalWords:  KeyFor containsKey java keyfor UnknownKeyFor KeyForBottom
%%  LocalWords:  2cm mapSet keySet km threeLetterWordSubset JT
